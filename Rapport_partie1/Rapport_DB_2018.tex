\documentclass{article}
\usepackage[utf8]{inputenc}
\usepackage{subfigure}
\usepackage{rotating}
\usepackage{nicefrac}
\usepackage{float}
\usepackage{amsmath}
\usepackage{amssymb}
\usepackage{hyperref}
\usepackage[export]{adjustbox}
\usepackage[T1]{fontenc}
\usepackage{enumitem}
\usepackage{lstautogobble}		% Pour mettre à gauche le code
\usepackage{listings}			% Pour afficher du code
\usepackage{graphicx} 			% Pour l'affichage d'images
\usepackage{titling} 			% Reduit l'espace au dessus du titre et auteurs
\usepackage[left=20mm,top=10mm, right=18mm, bottom=10mm, includefoot]{geometry}
\setlength{\droptitle}{-2em}   % Reduit l'espace au dessus du titre et auteurs

\title{\textsc{[INFO-H-303] Bases de données - Rapport Projet:}\
   "Ebay"}
\author{Alexandre Heneffe - 000440761\\
        Nicolas Jonas - 000}
\date{Mars 2018}

\begin{document}
\maketitle
\section{Introduction}

Ceci est le rapport du projet de base de données. Celui-ci contiendra le schéma entité-relationnel (et ses contraintes) de la future base de données ainsi que sa traduction en un modèle relationnel.

\section{Modèle Entité-Relationnel}

\subsection{Schéma}

\begin{center}
    \includegraphics[scale=0.45,left]{schemaEA}
\end{center}

\newpage

\subsection{Contraintes}

\begin{itemize} [label=\textbullet]
	\item La date de vente d'un objet doit être > à la date de mise en vente de cet objet
	\item Le prix proposé pour une proposition d'achat doit être >= au prix minimum de l'objet
	\item Un administrateur ne peut pas se supprimer lui-même
	\item L'évaluation d'un objet est faite que si cet objet est acheté depuis moins de 10 jours, c'est-à-dire que la date d'évaluation doit être < que la date de vente de l'objet+ 10
	\item La date d'évaluation doit être > que la date de vente d'un objet
	\item Le pseudo d'un utilisateur est unique (c'est ce qui permet de différencier les utilisateurs)
	\item Le vendeur d'un objet ne peut pas être l'acheteur de cet objet
	\item La date de naissance d'un vendeur doit être >= que 18
	\item la date de naissance d'un vendeur doit être > que la date de mise en vente d'un objet qu'il vend
	
\end{itemize}

\subsection{Hypothèses}

\begin{itemize}[label=\textbullet]
	\item Un objet appartient au minimum à 1 catégorie
	\item Une catégorie peut exister sans objet y étant associée
	\item Chaque objet est unique, il n'y a pas de notion de quantité
	\item Un utilisateur doit être majeur afin d'acheter des objets

\end{itemize}

\subsection{Justifications}
La généralisation (Utilisateur, vendeur et administrateur) dans le schéma, est partiel et couvrant.
\begin{itemize}

\item Partiel: Les vendeurs et administrateurs sont des types d'utilisateurs. Mais il y a aussi les utilisateurs classiques.

\item Couvrant: Un utilisateur peut être à la fois vendeur et administrateur par exemple.

\end{itemize}

\newpage 

\section{Modèle Relationnel}
Dans cette section, nous traduisons le modèle entité-relationnel de la section précédente en un modèle relationnel et en ajoutant ses contraintes.

\subsection{Modèle}

\indent

\paragraph{Objet} (\underline{Titre}, Description, DateMiseEnVente, PrixMin, Vendeur, DateVente, Acheteur, \underline{NumVendeur}, \underline{NumAdmin})\\
\indent - Objet.NumVendeur référence Vendeur.Numéro\\
\indent - Objet.NumAdmin référence Administrateur.Numéro


\paragraph{Vendeur} (\underline{Numero}, Nom, Prénom, DateNaissance, Adresse, \underline{PseudoUser})\\
\indent - Vendeur.PseudoUser référence Utilisateur.Pseudo


\paragraph{Administrateur} (\underline{Numero}, \underline{PseudoUser})\\
\indent - Administrateur.PseudoUser référence Utilisateur.Pseudo


\paragraph{Utilisateur} (\underline{Pseudo}, AddresseMail, MotDePasse, 
Description, \underline{NumAdmin})\\
\indent - Utilisateur.NumAdmin référence Administrateur.Numéro


\paragraph{PropositionAchat} (\underline{NuméroObjet}, \underline{Acheteur}, Prix, Etat, \underline{TitreObj}, \underline{PseudoUser})\\
\indent - PropositionAchat.TitreObj référence Objet.Titre\\
\indent - PropositionAchat.PseudoUser référence Utilisateur.Pseudo


\paragraph{Evaluation} (\underline{Numéro}, Note, Date, Commentaire, \underline{TitreObj}, \underline{PseudoVendeur}, \underline{PseudoUser})\\
\indent - Evaluation.TitreObj référence Objet.Titre\\
\indent - Evaluation.PseudoVendeur référence Vendeur.Pseudo\\
\indent - Evaluation.PseudoUser référence Utilisateur.Pseudo


\paragraph{Catégorie} (\underline{Titre}, Description, \underline{NumAdmin})\\
\indent - Catégorie.NumAdmin référence Administrateur.Numéro


\paragraph{Appartenance} (\underline{TitreObj, TitreCatégorie})\\
\indent - Appartenance.TitreObj référence Objet.Titre\\
\indent - Appartenance.TitreCatégorie référence Catégorie.Titre


\paragraph{Modifiation} (\underline{TitreCatégorie, NumAdmin})\\
\indent - Modification.TitreCatégorie référence Catégorie.Titre\\
\indent - Modification.NumAdmin référence Administrateur.Numéro


\subsection{Contraintes}



\end{document}